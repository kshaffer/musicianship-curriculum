\input{mmd-article-header}
\def\mytitle{Learning and mastery objectives for Musicianship sequence}
\def\myauthor{Kris P. Shaffer}
\def\latexmode{memoir}
\input{mmd-article-begin-doc}
The following are objectives that each student should accomplish by the end of the four-semester Musicianship course sequence.

\chapter{Holistic}
\label{holistic}

\section{Transcription (finished)}
\label{transcriptionfinished}

\begin{itemize}
\item Transcribe all vocal parts of a four-part Lutheran chorale setting.

\item Transcribe a typical pop\slash rock song: all vocal parts, all melodic instrumental lines (no block chords or strumming), and lead-sheet chord symbols for all harmonies.

\item Transcribe a nineteenth-century German art song: vocal part and piano.

\item Transcribe the slow movement of an 18th-century sonata-style piece for string quartet.

\item Transcribe a single movement of a short atonal or 12-tone chamber work (e.g., a movement of a Webern string quartet).

\end{itemize}

\section{Dictation (finished)}
\label{dictationfinished}

\begin{itemize}
\item Notate a four-bar-long, diatonic, non-modulating melody or bass line after a single hearing.

\item Notate a two-bar-long, non-modulating melody or bass line containing tonal chromatic elements after a single hearing.

\end{itemize}

\section{Aural recognition (unfinished)}
\label{auralrecognitionunfinished}

\begin{itemize}
\item Formal analysis by ear ***

\item Harmonic analysis by ear ***

\item Schema analysis by ear ***

\end{itemize}

\section{Sight-singing (finished)}
\label{sight-singingfinished}

\begin{itemize}
\item Sing from sight the melody to a contemporary Christian worship song.

\item Sing from sight any vocal part from a four-part Lutheran chorale.

\item Sing from sight the vocal part to an 18th-century German art song.

\item Sing from sight a vocal part from an atonal or 12-tone work.

\end{itemize}

\section{Improvisation (finished)}
\label{improvisationfinished}

\begin{itemize}
\item Improvise an alto or tenor line to accompany the melody and chords of a contemporary Christian worship song.

\end{itemize}

\section{Model composition (finished)}
\label{modelcompositionfinished}

\begin{itemize}
\item Compose a four-part Lutheran chorale based on a given melody.

\item Compose a contemporary Christian worship song (text, melody, alto, tenor, chords, bass line).

\item Compose a minuet for keyboard in the style of Joseph Haydn.

\item Compose a song to a German text in the style of Franz Schubert.

\end{itemize}

\section{Analysis (unfinished)}
\label{analysisunfinished}

\begin{itemize}
\item Correctly identify and label all keys, chords (with Roman numerals), embellishing tones, and cadences in a Lutheran chorale.

\item Correctly identify and label all keys, chords (with Roman numerals), embellishing tones, and cadences in a pop\slash rock song.

\item Correctly identify and label all keys, chords (with Roman numerals and functional bass), embellishing tones, and cadences in an eighteenth-century keyboard work.

\item text and music

\end{itemize}

\section{Writing (unfinished)}
\label{writingunfinished}

\begin{itemize}
\item Generate original thesis based on musical analysis.

\item Construct clear linear argument in support of a thesis and devoid of unnecessary details.

\item Use musical terminology appropriately and effectively.

\item Demonstrate mastery of \emph{Chicago Manual of Style} for writing on music and citing sources.

\end{itemize}

\section{Software and technology (unfinished)}
\label{softwareandtechnologyunfinished}

\chapter{Content}
\label{content}

\section{Fundamentals (unfinished)}
\label{fundamentalsunfinished}

\begin{itemize}
\item identify standard meters by ear from recording (simple and compound; duple, triple, and quadruple meters)

\item identify bar-long rhythmic patterns by ear from piano performance (durations from bar-length to division-length)

\item Sing major scales

\item Sing minor scales (natural, harmonic, melodic)

\item identify major and minor scales by ear from piano

\item identify major and minor mode by ear from recording

\item identify pitches on the treble staff

\item write pitches on the treble staff

\item identify pitches on the bass staff

\item write pitches on the bass staff

\item identify pitches on the alto staff

\item write pitches on the alto staff

\item identify rhythmic durations from standard rhythmic notation

\item identify metric placement from standard rhythmic notation

\item write rhythmic notation

\item identify standard meters from a time signature (simple and compound; duple, triple, and quadruple meters)

\item write appropriate time signatures for given meters and beat values

\item identify major key signatures on the treble \& bass staves

\item identify minor key signatures on the treble \& bass staves

\item write major key signatures on the treble \& bass staves

\item write minor key signatures on the treble \& bass staves

\item identify major scales on the treble \& bass staves

\item identify minor scales on the treble \& bass staves

\item write major scales on the treble, alto, and bass staves

\item write minor scales on the treble, alto, and bass staves

\item identify chromatic scales on the treble, alto, and bass staves

\item write chromatic scales on the treble, alto, and bass staves

\end{itemize}

\begin{quote}

still need intervals, triads, seventh chords, types of motion
\end{quote}

\section{Transcription (finished)}
\label{transcriptionfinished}

\begin{quote}

Optional shorter exercises. Otherwise, see holistic transcription projects.
\end{quote}

\section{Dictation (unfinished)}
\label{dictationunfinished}

\begin{itemize}
\item Dictate a four-bar diatonic melody in simple or compound melody from a single hearing.

\item Dictate a four-bar modulating diatonic melody in simple or compound melody from a single hearing.

\item Dictate a four-bar diatonic melody with chromatic elements in simple or compound melody after two hearings.

\end{itemize}

\section{Aural recognition (unfinished)}
\label{auralrecognitionunfinished}

\begin{itemize}
\item Recognize standard pop\slash rock harmonic schemata from a single hearing. *** \emph{list?} ***

\item Recognize standard contrapuntal and embellishing devices from a single hearing. *** \emph{list?} ***

\item Recognize stock prolongational and cadential patterns from a single hearing. *** \emph{list?} ***

\end{itemize}

\begin{quote}

playlist construction
\end{quote}

\section{Performance (unfinished - lots more singing)}
\label{performanceunfinished-lotsmoresinging}

\begin{itemize}
\item Play major scales at the keyboard (one octave, any tempo)

\item Play minor scales at the keyboard (one octave, any tempo)

\item Play Fuxian cantus firmus at the keyboard.

\item Sing 4-bar rhythm from \emph{protonotation} (durations from bar-length to division-length)

\item Sing 4-bar melody from \emph{solfège} syllables in major that starts and ends on \emph{do} and moves in all stepwise motion

\item Sing 4-bar melody from \emph{solfège} syllables in minor that starts and ends on \emph{do} and moves in all stepwise motion

\item Sing 4-bar melody from \emph{protontation} in major that starts and ends on \emph{do}, moves in all stepwise motion, and contains rhythmic durations from the bar length to the division length

\item Sing 4-bar melody from \emph{protontation} in minor that starts and ends on \emph{do}, moves in all stepwise motion, and contains rhythmic durations from the bar length to the division length

\item Sing 4-bar melody from \emph{protontation} in major that starts and ends on \emph{do}, moves in all stepwise motion or leaps between members of the tonic triad (\emph{do}, \emph{mi}, \emph{sol}), and contains rhythmic durations from the bar length to the division length

\item Sing 4-bar melody from \emph{protontation} in minor that starts and ends on \emph{do}, moves in all stepwise motion or leaps between members of the tonic triad (\emph{do}, \emph{me}, \emph{sol}), and contains rhythmic durations from the bar length to the division length

\item Sing 4-bar rhythm from standard rhythmic notation (durations from bar-length to division-length)

\item Sing 4-bar melody from standard pitch notation in major that starts and ends on \emph{do} and moves in all stepwise motion

\item Sing 4-bar melody from standard pitch notation in minor that starts and ends on \emph{do} and moves in all stepwise motion

\item Sing 4-bar melody from standard musical notation in major that starts and ends on \emph{do}, moves in all stepwise motion, and contains rhythmic durations from the bar length to the division length

\item Sing 4-bar melody from standard musical notation in minor that starts and ends on \emph{do}, moves in all stepwise motion, and contains rhythmic durations from the bar length to the division length

\item Sing 4-bar melody from standard musical notation in major that starts and ends on \emph{do}, moves in all stepwise motion or leaps between members of the tonic triad (\emph{do}, \emph{mi}, \emph{sol}), and contains rhythmic durations from the bar length to the division length

\item Sing 4-bar melody from standard musical notation in minor that starts and ends on \emph{do}, moves in all stepwise motion or leaps between members of the tonic triad (\emph{do}, \emph{me}, \emph{sol}), and contains rhythmic durations from the bar length to the division length

\end{itemize}

\section{Improvisation (finished)}
\label{improvisationfinished}

\begin{itemize}
\item Improvise a (sung) melodic line that fits Fux's guidelines for a cantus firmus.

\item Improvise a (sung) first-species counterpoint line to a cantus firmus sung by a partner.

\item Improvise a (sung) first-species counterpoint while playing a cantus firmus at the keyboard.

\item Improvise a (sung) second-species counterpoint line to a cantus firmus sung by a partner.

\item Improvise a (sung) second-species counterpoint while playing a cantus firmus at the keyboard.

\item Improvise a (sung) third-species counterpoint line to a cantus firmus sung by a partner.

\item Improvise a (sung) third-species counterpoint while playing a cantus firmus at the keyboard.

\item Improvise a (sung) fourth-species counterpoint line to a cantus firmus sung by a partner.

\item Improvise a (sung) fourth-species counterpoint while playing a cantus firmus at the keyboard.

\end{itemize}

\begin{quote}

additional practice work: in-class practice singing alto and tenor lines with a contemporary worship song melody sung by a partner or played at the keyboard.
\end{quote}

\section{Model composition (finished)}
\label{modelcompositionfinished}

\begin{itemize}
\item Compose two flawless first-species counterpoints above a cantus firmus.

\item Compose two flawless first-species counterpoints below a cantus firmus.

\item Compose two flawless second-species counterpoints above a cantus firmus.

\item Compose two flawless second-species counterpoints below a cantus firmus.

\item Compose two flawless third-species counterpoints above a cantus firmus.

\item Compose two flawless third-species counterpoints below a cantus firmus.

\item Compose two flawless fourth-species counterpoints above a cantus firmus.

\item Compose two flawless fourth-species counterpoints below a cantus firmus.

\item Compose alto, tenor, and bass for three non-modulating Lutheran-chorale-style phrases.

\item Compose alto, tenor, and bass for two modulating Lutheran-chorale-style phrases.

\item Compose alto and tenor vocal parts to accompany the melody and chords of two contemporary Christian worship songs.

\item Compose a minuet melody over a bass line.

\item Compose two melodies for German romantic poem texts.

\end{itemize}

\begin{quote}

additional practice work: in-class practice identifying \emph{galant} schemas from bass lines and composing appropriate melodies; writing piano parts for German lied melodic phrases.
\end{quote}

\section{Analysis (unfinished)}
\label{analysisunfinished}

\begin{quote}

reduction
\end{quote}

\section{Writing (unfinished)}
\label{writingunfinished}

\section{Software and technology (unfinished)}
\label{softwareandtechnologyunfinished}

\chapter{Licensing}
\label{licensing}

This work is copyright ©2012 by Kris P. Shaffer and is licensed under a Creative Commons Attribution 3.0 Unported License: http:/\slash creativecommons.org\slash licenses\slash by\slash 3.0\slash .

\input{mmd-memoir-footer}

\end{document}
